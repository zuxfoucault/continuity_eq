\documentclass[paper=a4,fontsize=12pt,pagesize,DIV=calc]{scrreprt}
\input{qnote}
\renewcommand*\theequation{\arabic{equation}}


\begin{document}

\noindent The first-order ordinary differential equation (ODE)
\begin{equation}
\frac{d}{dt} x_t = v_t(x_t) \label{eq:ode}
\end{equation}
where $x_t \in \mathbb{R}^d$ for time $t \in [0, T]$, and where $v : [0, T] \times \mathbb{R}^d \to \mathbb{R}^d$ is a time-dependent vector field that is generally non-linear. If we sample a random initial condition $x_0 \sim p_0$ from some distribution $p_0$, then the time-evolved random vector $x_t \sim p_t$ will also be a random variable with some probability density function (PDF) $p_t$, for all $t \in [0, T]$. By invoking the change of variable formula for PDFs on infinitesimal time steps of Eq. (\ref{eq:ode}), one can show that $p_t$ satisfies the \emph{continuity equation}
\begin{equation}
\frac{\partial}{\partial t} p_t(x)
= - \nabla \cdot \bigl[v_t(x)p_t(x)\bigr] \label{eq:ce}
\end{equation}
with respect to the velocity field $v_t$, for time $t \in [0, T]$.

\begin{proof}
Fix $t$ and consider a small time increment $\varepsilon > 0$. By a first-order Taylor expansion of the ODE solution,
\begin{equation}
x_{t+\varepsilon}
= x_t + \varepsilon v_t(x_t) + o(\varepsilon). \label{eq:increment}
\end{equation}
Define the deterministic map
\begin{equation}
T_\varepsilon(x) := x + \varepsilon v_t(x), \label{eq:deter}
\end{equation}
so that, to first order in $\varepsilon$, we have $x_{t+\varepsilon} = T_\varepsilon(x_t)$.

\noindent Since $x_{t+\varepsilon}$ is obtained by pushing forward $x_t$ through $T_\varepsilon$, the change-of-variables formula for probability densities yields
\begin{equation}
p_{t+\varepsilon}(y)
= p_t\bigl(T_\varepsilon^{-1}(y)\bigr)
\left|\det \nabla T_\varepsilon^{-1}(y)\right|. \label{eq:cov}
\end{equation}

\noindent From the definition of $T_\varepsilon$, its inverse admits the expansion
\begin{equation}
T_\varepsilon^{-1}(y)
= y - \varepsilon v_t(y) + o(\varepsilon). \label{eq:inv}
\end{equation}
Therefore (Taylor expansion):
\begin{equation}
p_t\bigl(T_\varepsilon^{-1}(y)\bigr)
= p_t(y) - \varepsilon \nabla p_t(y) \cdot v_t(y) + o(\varepsilon). \label{eq:taylor}
\end{equation}

\noindent The Jacobian matrix of $T_\varepsilon$ is
\begin{equation}
\nabla T_\varepsilon(x) = I + \varepsilon \nabla v_t(x), \label{eq:jacobian}
\end{equation}
so that
\begin{equation}
\det \nabla T_\varepsilon(x)
= 1 + \varepsilon \nabla \cdot v_t(x) + o(\varepsilon). \label{eq:jacobian-determinant}
\end{equation}
Consequently,
\begin{equation}
\det \nabla T_\varepsilon^{-1}(y)
= 1 - \varepsilon \nabla \cdot v_t(y) + o(\varepsilon). \label{eq:j2}
\end{equation}
Substituting the above expansions into the change-of-variables formula, we obtain
\begin{align}
p_{t+\varepsilon}(y)
&= \Bigl(p_t(y) - \varepsilon \nabla p_t(y) \cdot v_t(y)\Bigr)
   \Bigl(1 - \varepsilon \nabla \cdot v_t(y)\Bigr)
   + o(\varepsilon) \\
&= p_t(y)
   - \varepsilon \Bigl[
      \nabla p_t(y) \cdot v_t(y)
      + p_t(y)\nabla \cdot v_t(y)
     \Bigr]
   + o(\varepsilon). \label{ep:cov}
\end{align}
Noting that
\begin{equation}
\nabla \cdot \bigl(v_t(y)p_t(y)\bigr)
= \nabla p_t(y) \cdot v_t(y)
  + p_t(y)\nabla \cdot v_t(y), \label{eq:product}
\end{equation}
we conclude that
\begin{equation}
p_{t+\varepsilon}(y)
= p_t(y) - \varepsilon \nabla \cdot \bigl(v_t(y)p_t(y)\bigr)
  + o(\varepsilon). \label{eq:cld}
\end{equation}

\noindent Dividing both sides by $\varepsilon$ and taking the limit $\varepsilon \to 0$ yields
\begin{equation}
\frac{\partial}{\partial t} p_t(y)
= - \nabla \cdot \bigl(v_t(y)p_t(y)\bigr), \label{eq:limit}
\end{equation}
which is precisely the continuity equation.
\end{proof}

\end{document}
